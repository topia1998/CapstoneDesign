\clearpage
\section {Fruit Shopping\hfill 8 Points}
Sogang-Mart asks SogangCS to develop a following CLI program for fruit shopping.
In the program, keys from 0 to 5 represents fruits to be added.
If a user presses one of them, the selected fruit is added to the basket and the total price is accumulated. 
If the program receives key 0, it prints the total price and terminates the program.
Our task is to check the consistency of the program.

\begin{figure}[h]
\begin{lstlisting}[language=C]
#include <stdio.h>
int main() {
    char choice;
    int count = 0;
    int total = 0;
    printf("1. Add one apple to your basket.\n");
    printf("2. Add one banana to your basket.\n");
    printf("3. Add one orange to your basket. \n");
    printf("4. Add one lemon to your basket. \n");
    printf("5. Add one grape to your basket. \n");
    printf("0. Compute the total price and exit\n");
    while (1) {
        printf("Select Number (0~5): ");
        scanf(" %c", &choice);
        switch (choice) {
            case '0':
                printf("Total price is %d.\n", total);
                printf("program finished.\n");
            case '1':
                printf("Apple is selected. Price: 1000\n");
                total += 1000;
            case '2':
                printf("Banana is selected. Price: 1500\n");
                total += 1500;
            case '3':
                printf("Oragne is selected. Price: 2000\n");
                total += 2000;
            case '4':
                printf("Lemon is selected. Price: 2500\n");
                total += 2500;
            case '5':
                printf("Grape is selected. Price: 3000\n");
                total += 3000;
            default:
                printf("Wrong input. Type between 0 to 5.\n");
                break;
         }  
        if (choice == '0') break;
    }
    return 0;
}
\end{lstlisting}
\caption{A CLI code of fruit shopping}
\end{figure}
\begin{enumerate}
\item Write reasons for a space between " and \%c" in line 18. \hfill 2 Points
\begin{solution}{5cm}
To clear '\textbackslash n' in input buffer.
\end{solution}
\clearpage
\item If the program receives the stdin input shown in Figure~\ref{cli_input}, write the response of the program for each input. \hfill 2 Points
\newline
\begin{figure}[h]
\centering
\begin{varwidth}{\linewidth}
\begin{verbatim}
Select Number (0~5): 1
Select Number (0~5): 0
\end{verbatim}
\end{varwidth}
\caption{Stdin input}
\label{cli_input}
\end{figure}
\begin{solution}{5cm}
Apple is selected. Price: 1000\newline
Banana is selected. Price: 1500\newline
Oragne is selected. Price: 2000\newline
Lemon is selected. Price: 2500\newline
Grape is selected. Price: 3000\newline
Wrong input. Type between 0 to 5.\newline
Total price is 10000.\newline
program finished.\newline
Apple is selected. Price: 1000\newline
Banana is selected. Price: 1500\newline
Oragne is selected. Price: 2000\newline
Lemon is selected. Price: 2500\newline
Grape is selected. Price: 3000\newline
Wrong input. Type between 0 to 5.\newline
\end{solution}
\item Is the code working correctly? If not, find bugs and fix them.  \hfill 2 Points
\begin{solution}{5cm}
There is no break statement for each case body.
Add break statement in every case body. \newline\newline
\end{solution}
\item Write output of the fixed program with the input in Figure~\ref{cli_input2}. \hfill 2 Points
\begin{figure}[h]
\centering
\begin{varwidth}{\linewidth}
\begin{verbatim}
Select Number (0~5): 1
Select Number (0~5): 2
Select Number (0~5): 3
Select Number (0~5): 4
Select Number (0~5): 5
Select Number (0~5): 0
\end{verbatim}
\end{varwidth}
\caption{Stdin input}
\label{cli_input2}
\end{figure}
\begin{solution}{5cm}
Apple is selected. Price: 1000\newline
Banana is selected. Price: 1500\newline
Oragne is selected. Price: 2000\newline
Lemon is selected. Price: 2500\newline
Grape is selected. Price: 3000\newline
Total price is 10000.
program finished.
\end{solution}
\end{enumerate}
\clearpage